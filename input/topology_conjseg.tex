\section{Conjugated segments}
\label{sec:xmlsegments}

The file describing hopping sites, or conjugated segments, is used by practically all programs and calculators. It links the coarse-grained trajectory (positions and orientations of rigid fragments) and quantum-mechanical descriptions of all conjugated segments. The description of this \xml file (\xmlsegments) is given in table \ref{tab:segments}. An example for \dcvt is shown in listing~\ref{list:segments}.

\begin{table}[h]
\caption{Description of conjugated segments (\xmlsegments).} 
\label{tab:segments}
\rowcolors{1}{invisiblegray}{white} {\small \input{reference/xml/segments.xml} }
\end{table}

\lstset{
  language=XML,
  frame=lines,
  basicstyle=\ttfamily\footnotesize,
  identifierstyle=\color{red},
  keywordstyle=\color{blue},
  showstringspaces=false,
  columns=fullflexible,
  commentstyle=\color{gray}\rmfamily\itshape,
  morekeywords={segments,segment,coordinates,orbitals,basisset,torbital,reorganization,qneutral,qcharged,energy,beadconj,molname,name,map,weights},
}

\lstinputlisting[
 label=list:segments, 
 caption={\small \xml file describing \slink{segments}{conjugated segments}. Note that the mapping and weights for each segment are separated by a colon. 
}]%
{./input/segments.xml}


