\section{Mapping file}
\label{sec:xmlmap}

The mapping file (referred here as \xmlcsg) is used by the program \ctpmap to convert an atomistic trajectory to a coarse-grained one. The coarse-grained trajectory contains positions, names, types, and orientations of rigid fragments. \xmlcsg contains definitions of rigid fragments (coarse-grained beads) and identifies to what conjugated segment a particular rigid fragment belongs. The description of the mapping options is given in table \ref{tab:map}. An example of \xmlcsg for a \dcvt molecule is shown in listing~\ref{list:map}. 

\begin{table}[ht]
\caption{Description of the \xml mapping file.}
\label{tab:map}
\rowcolors{1}{invisiblegray}{white} {\small \input{reference/xml/map.xml}}
\end{table}

\vfill

\lstinputlisting[
 language=XML,
 basicstyle=\ttfamily\scriptsize,
 commentstyle=\color{gray}\ttfamily,
 label=list:map, 
 morekeywords={cg_molecule,cg_beads,cg_bead,crgunitname,bead,beads,type,topology,name,ident,maps,map,mapping,weights,position,qm,symmetry},
 caption={Examle of \xmlcsg for \dcvt. Each rigid fragment (coarse-grained bead) is defined by a list of atoms. Atom numbers, names, and residue names should correspond to those used in \gromacs topology (see the corresponing listing \ref{list:pdb} of the pdb file).}]%
{./input/map.xml}

\vfill