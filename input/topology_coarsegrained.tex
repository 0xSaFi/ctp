\section{Mapping file}
\label{sec:xmlmap}
The mapping file (referred here as \xmlcsg) is used by the program \ctpmap to convert an atomistic trajectory to a trajectory with conjugated segments and rigid fragments. 
This trajectory is stored in a \slink{statefile}{state file} and contains positions, names, types of atoms belonging to rigid fragments. 
The description of the mapping options is given in table \ref{tab:map}. An example of \xmlcsg for a \dcvt molecule is shown in listing~\ref{list:map}. 

The file \xmlcsg contains the whole electrostatic information about the molecules as well as the structural information. The \tool{pdb2map} creates a \xmlcsg from a pdb file and is a good starting point for further refinement. 
%
\begin{table}[h]
\label{tab:map}
\caption{Description of the \xml mapping file (\xmlcsg).}
\rowcolors{1}{invisiblegray}{white} {\footnotesize \input{reference/xml/map.xml}}
\end{table}
%
% Define new language for listings.
\lstdefinelanguage{MXML} {
   basicstyle=\ttfamily\scriptsize,
   sensitive=true,
   morecomment=[s][\color{gray}\rmfamily\itshape]{<!--}{-->}, 
   showstringspaces=false,
   numberstyle=\scriptsize,
   numberblanklines=true,
   showspaces=false,
   breaklines=true,
   showtabs=false,
   alsoletter={:},
   keywords = [1]
   { topology,molecules,molecule,name,mdname,segments,segment,fragments,fragment,mdatoms,qmatoms,localframe,weights},
   keywordstyle={[1]\color{blue}},
}

\lstinputlisting[
 language=MXML,
 label=list:map,
 caption={Examle of \xmlcsg for \dcvt. Each rigid fragment (coarse-grained bead) is defined by a list of atoms. Atom numbers, names, and residue names should correspond to those used in \gromacs topology (see the corresponing listing \ref{list:pdb} of the pdb file).}]%
{./input/dcv2t/map.xml}
