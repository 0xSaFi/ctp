\chapter{Site energies}
\label{sec:site_energies}

\newcommand{\estat}{\hyperref[calc:estat]{\texttt{estat}}\xspace}

\section{Electrostatic energy}
\label{sec:electrostatic}
Variations of the local electric field can result in large electrostatic contributions to the energetic disorder. Using the atomic partial charges of charged and neutral molecules, $\Delta E_{ij}^\text{el}$ can be computed from the site energies~\cite{kirkpatrick_columnar_2008}
\begin{equation}
E_{i}^\text{el}  = \frac{1}{4 \pi \epsilon_0} \sum_{a_i} \sum_{\substack{b_k   \\ k\neq i }}
\frac{ \left( q^c_{a_i} - q^n_{a_i} \right) q^n_{b_k}}{ \epsilon_\text{s} r_{a_i b_k}} 
\, ,
\label{equ:estatic}
\end{equation}
where $r_{a_i b_k}=|\vec{r}_{a_i} - \vec{r}_{b_k}|$ is the distance between atoms $a_i$ and $b_k$,   $\epsilon_\text{s}$ is the static relative dielectric constant.
%
The first sum extends over all atoms of molecule $i$, for which the site energy is calculated. The second sum reflects interactions with all atoms of neutral molecules $k \ne i$. By using \equ{estatic}, one assumes that the influence of conformational changes on partial charges and changes of the molecular geometry upon charging are small. In order to minimize finite size effects, we do not use spherical cutoff but apply the nearest image convention, that is sum over all neutral molecules in the box after centering the box around the charged molecule. 

The influence of polarization effects on the Coulomb interactions can be taken into account by using a relative dielectric constant in \equ{estatic}. Bulk values of  $\epsilon_\text{s} = 2-5$ for typical organic semiconductors uniformly scale all site energies but are not capable of describing polarization effects on a microscopic level. 
The contribution to $E_i^\text{el}$ from the first coordination shell is then underestimated due to overscreening and, as a result, the site-energy differences become artificially small. Alternatively, one can introduce a phenomenological distance-dependent screening function $\epsilon(r_{a_i b_k})$ in~\equ{estatic}~\cite{nagata_atomistic_2008}
\begin{equation}
\epsilon(r)=\epsilon_{\text{s}} - (\epsilon_{\text{s}} - 1)
\left( 1 + sr + \frac{1}{2}s^2r^2 \right) 
\mathrm{e}^{ -sr}\,,
\label{equ:epss}
\end{equation}
where the parameter $s$ is the inverse screening length. For a monovalent ion in water, for example, $\epsilon_{\text{s}}=80$ and $s=3\,\textrm{nm}^{-1}$~\cite{daggett_molecular_1991}. This screening function ensures that neighboring atoms interact via an unscreened Coulomb potential ($\epsilon \sim 1$) while the electrostatic interaction between atoms at large separations is screened as in the bulk. 

Evaluation of the electrostatic contribution is provided by the \estat calculator of \texttt{ctp\_run} 
\begin{verbatim}
  ctp_run --exec "estat"
\end{verbatim}
\suggestion{Need to describe all possible options, with and without screening, etc}