\chapter{Electronic couplings}
\label{sec:transfer_integrals}

The electronic transfer integral\index{electronic coupling}\index{transfer integral|see{electronic coupling}} element $J_{ij}$ entering the Marcus rates in \equ{marcus} is defined as
\begin{equation}
   J_{ij} = \left\langle \phi_i \left\vert \hat{H} \right\vert \phi_j \right\rangle ,
\label{equ:TI}
\end{equation}
where $\phi_i$ and $\phi_j$ are diabatic wavefunctions, localized on molecule $i$ and $j$ respectively, participating in the charge transfer, and $\hat{H}$ is the Hamiltonian of the formed dimer. Within the frozen-core approximation, the usual choice for the diabatic wavefunctions $\phi_i$\index{diabatic states} is the highest occupied molecular orbital (HOMO) in case of hole transport, and the lowest unoccupied molecular orbital (LUMO) in the case of electron transfer, while $\hat{H}$ is an effective single particle Hamiltonian, e.g. Fock or Kohn-Sham operator of the dimer. As such, $J_{ij}$ is a measure of the strength of the electronic coupling of the frontier orbitals of monomers mediated by the dimer interactions. 

Intrinsically, the transfer integral is very sensitive to the molecular arrangement, i.e. the distance and the mutual orientation of the molecules participating in charge transport. Since this arrangement can also be significantly influenced by static and/or dynamic disorder~\cite{baessler_charge_1993,troisi_charge-transport_2006,troisi_charge_2009,mcmahon_organic_2010,vehoff_charge_2010},
it is essential to calculate $J_{ij}$ explicitly for each hopping pair within a realistic morphology. Considering that the number of dimers for which \equ{TI} has to be evaluated is proportional to the number of molecules times their coordination number, computationally efficient and at the same time quantitatively reliable schemes are required.

\subsection{Projection of monomer orbitals on dimer orbitals (DIPRO)}
\label{sec:dipro}
An approach for the determination of the transfer integral that can be used for any single-particle electronic structure method (Hartree-Fock, DFT, or semiempirical methods) is based on the projection of monomer orbitals on a manifold of explicitly calculated dimer orbitals. This dimer projection (DIPRO) technique including an assessment of computational parameters such as the basis set, exchange-correlation functionals, and convergence criteria is presented in detail in Ref.~\cite{baumeier_density-functional_2010}. A brief summary of the concept is given below.

We start from an effective Hamiltonian~\footnote{we use following notations: $a$ - number, $\vctr{a}$ - vector, $\matr{A}$ - matrix, $\oper{A}$ - operator}
%
\begin{equation}
  \oper{H}^\text{eff} = \sum_i \epsilon_i \oper{a}_i^\dagger \oper{a}_i + \sum_{j \neq i} J_{ij} \oper{a}_i^\dagger \oper{a}_j + c.c.
  \label{equ:dipro_eq1}
\end{equation}
%
where $\oper{a}_i^\dagger$ and $\oper{a}_i$ are the creation and annihilation operators for a charge carrier located at the molecular site $i$.
The electron site energy is given by $\epsilon_i$, while $J_{ij}$  is the transfer integral between two sites $i$ and $j$. We label their frontier orbitals (HOMO for hole transfer, LUMO for electron transfer) $\phi_i$ and $\phi_j$, respectively. Assuming that the frontier orbitals of a dimer (adiabatic energy surfaces) result exclusively from the interaction of the frontier orbitals of monomers, and consequently expand them in terms of $\phi_i$ and $\phi_j$. The expansion coefficients, $\vctr{C}$, can be determined by solving the secular equation
%
\begin{equation}
  (\matr{H} - E \matr{S})\vctr{C} = 0
  \label{equ:dipro_eq2}
\end{equation}
%
where $\matr{H}$ and $\matr{S}$ are the Hamiltonian and overlap matrices of the system, respectively. 
%
%Since it is easier to work in matrix form, the following
%equation also holds (equation (\ref{eq:dipro_eq2}) in matrix form):
%
%\begin{equation}
% \matr{H}\matr{U} = \matr{S}\matr{U}\matr{E}
%  \label{eq:dipro_eq3}
% \end{equation}
%
These matrices can be written explicitly as
%
\begin{equation}
% \begin{aligned}
  \matr{H} = 
  \begin{pmatrix}
    e_i    &  H_{ij} \\
    H_{ij}^* &  e_j  
  \end{pmatrix} \hspace{2cm}
  \matr{S} = 
  \begin{pmatrix}
    1    &  S_{ij} \\
    S_{ij}^* &  1  
  \end{pmatrix}
%  \end{aligned}
  \label{equ:dipro_eq3}
\end{equation}
%
with 
%
\begin{equation}
 \begin{aligned}
  e_i &= \Bra{\phi_i}\oper{H} \Ket{\phi_i} \hspace{2cm}  H_{ij} = \Bra{\phi_i}\oper{H} \Ket{\phi_j}\\
  e_j &= \Bra{\phi_j}\oper{H} \Ket{\phi_j} \hspace{2cm}  S_{ij} = \Bra{\phi_j} \phi_j\rangle %S 
 \end{aligned}
  \label{equ:dipro_eq4}
\end{equation}
The matrix elements $e_{i(j)}$, $H_{ij}$, and $S_{ij}$ entering \equ{dipro_eq3} can be calculated via projections on the dimer orbitals (eigenfunctions of $\hat{H}$) $\left\{\Ket{\phi^\text{D}_n}\right\}$ by inserting $\oper{1} = \sum_n \Ket{\phi^\text{D}_n}\Bra{\phi^\text{D}_n}$ twice. We exemplify this explicitly for $H_{ij}$ in the following
%
\begin{equation}
  H_{ij} = \sum_{nm}{\Braket{\phi_i|\phi^\text{D}_n} \Bra{\phi^{D}_n}\hat{H}\Ket{\phi^\text{D}_m}\Braket{\phi^\text{D}_m|\phi_j}} .
  \label{eq:dipro_eq16}
\end{equation}
%
The Hamiltonian is diagonal in its eigenfunctions, $\Bra{\phi^\text{D}_n}\oper{H}\Ket{\phi^\text{D}_m} = E_n \delta_{nm}$. Collecting the projections of the frontier orbitals  $\Ket{\phi_{i(j)}}$ on the $n$-th dimer state $\left(\vctr{V}_i\right)_n= \Braket{\phi_i|\phi^\text{D}_n}$ and $\left(\vctr{V}_j\right)_n=\Braket{\phi_j|\phi^\text{D}_n}$ respectively, into vectors we obtain

\begin{equation}
   H_{ij} = \vctr{V}_i \matr{E}   \vctr{V}_j^\dagger .
  \label{eq:dipro_eq17}
\end{equation}
%
What is left to do is determine these projections $\vctr{V}_{i(j)}$. In all practical calculations the molecular orbitals are expanded in basis sets of either plane waves or of localized atomic orbitals $\Ket{\varphi_\alpha}$. We will first consider the case that the calculations for
the monomers are performed using a counterpoise basis set that is commonly used to deal with the basis set superposition error (BSSE). The basis set of atom-centered orbitals of a monomer is extended to the one of the dimer by adding the respective atomic orbitals at virtual coordinates of the second monomer. We can then write the respective expansions as

\begin{equation}
 %\begin{aligned}
  \Ket{\phi_{k}} = \sum_{\alpha} \lambda^{(k)}_\alpha \Ket{\varphi_\alpha} \hspace{1cm}\text{and}\hspace{1cm}
  \Ket{\phi^\text{D}_n} = \sum_{\alpha} D^{(n)}_\alpha \Ket{\varphi_\alpha}
  \label{eq:dipro_eq18}
\end{equation}
%
where $k=i,j$. The projections can then be determined within this common basis set as

 \begin{equation}
  \begin{aligned}
     \left(\vctr{V}_k\right)_n=\Braket{\phi_k|\phi^\text{D}_n} = \sum_{\alpha} \lambda^{(k)}_{\alpha} \Bra{\alpha} \sum_{\beta} D^{(n)}_{\beta} \Ket{\beta} = 
     \vctr{\boldsymbol{\lambda}}_{(k)}^\dagger \matr{\mathcal{S}} \vctr{D}_{(n)} 
%     %\\
% %    \Braket{B|i} = \sum_{\alpha} B_{\alpha} \Bra{\alpha}
% %    \sum_{\beta} D^{(i)}_{\beta} \Ket{\beta} = 
% %    \vctr{B}^\dagger \matr{S} \vctr{D}^{(i)} \\
  \end{aligned}
   \label{eq:dipro_eq19}
 \end{equation}
where $\matr{\mathcal{S}}$ is the overlap matrix of the atomic basis functions. This allows us to finally write the elements of the Hamiltonian and overlap matrices in \equ{dipro_eq3} as:

 \begin{equation}
  \begin{aligned}
     H_{ij} &= \vctr{\boldsymbol{\lambda}}_{(i)}^\dagger \matr{\mathcal{S}} \matr{D} \matr{E} \matr{D}^\dagger \matr{\mathcal{S}}^\dagger \vctr{\boldsymbol{\lambda}}_{(j)}  \\
     S_{ij} &= \vctr{\boldsymbol{\lambda}}_{(i)}^\dagger \matr{\mathcal{S}} \matr{D}  \matr{D}^\dagger \matr{\mathcal{S}}^\dagger \vctr{\boldsymbol{\lambda}}_{(j)} 
  \end{aligned}
   \label{eq:dipro_eq20}
 \end{equation}
%
Since the two monomer frontier orbitals that form the basis of this expansion are not orthogonal in general ($\matr{S} \neq \matr{1}$), it is necessary to transform \equ{dipro_eq2} into a standard eigenvalue problem of the form
%
\begin{equation}
  \matr{H}^{\mathrm{eff}} \vctr{C}^{\mathrm{eff}} =   E \vctr{C}^{\mathrm{eff}} 
  \label{eq:dipro_eq7}
\end{equation}
%
to make it correspond to \equ{dipro_eq1}. According to L\"owdin such a transformation can be achieved by
%
\begin{equation}
  \matr{H^\mathrm{eff}} = \matr{S}^{\left. {-1} \middle/ {2} \right.}
  \matr{H}\matr{S}^{\left. {-1} \middle/ {2} \right.}.
  \label{eq:dipro_eq9}
\end{equation}
%
This then yields an effective Hamiltonian matrix in an orthogonal basis, and its entries can directly be identified with the site energies $\epsilon_i$ and transfer integrals $J_{ij}$:
%
\begin{equation}
 \begin{aligned}
  \matr{H}^{\mathrm{eff}} &= 
    \begin{pmatrix}
      e_i^{\mathrm{eff}}    &  H_{ij}^\mathrm{eff} \\
      H_{ij}^{*,\mathrm{eff}}   &  e_j^\mathrm{eff}  
    \end{pmatrix} =
    \begin{pmatrix}
      \epsilon_i    &  J_{ij} \\
      J_{ij}^*      &  \epsilon_j  
    \end{pmatrix} 
 \end{aligned}
  \label{eq:dipro_eq11}
\end{equation}

% \begin{figure}[h]
%     \center
%     \includegraphics[width=0.8\textwidth]{fig/idft_flow/scheme_t}
%     \caption{DIPRO scheme}
%     \label{fig:scheme}
% \end{figure}



\subsection{Density-functional methods}
\label{sec:dft}
\index{electronic coupling!DFT}
While the use of the semiempirical ZINDO method provides an efficient on-the-fly technique to determine electronic coupling elements, it is not generally applicable to all systems. For instance, its predictive capacity with regards to atomic composition and localization behavior of orbitals within more complex structures is reduced. Moreover, transition- or semi-metals are often not even parametrized. In this case, {\it ab-initio} based approaches, e.g., density-functional theory can remedy the situation~\cite{huang_intermolecular_2004,huang_validation_2005,valeev_effect_2006,yin_balanced_2006,yang_theoretical_2007,baumeier_density-functional_2010}. Within the dimer projection method described in detail in Ref.~\cite{baumeier_density-functional_2010}, explicit quantum-chemical calculations are required for every molecule and every hopping pair in the morphology. As a consequence, this procedure is significantly more computationally demanding. The code currently contains scripts which support evaluation of transfer integrals from quantum-chemical calculations performed with the \gaussian and \turbomole packages.

Apart from semi-empirical methods, we also provide interfaces for a DFT-based evaluation of electronic coupling elements. The interfacing procedure consists of three main steps: generation of directory structures containing input coordinates for monomers and dimers, performing the actual quantum-chemical calculations and calculating the transfer integrals using the DIPRO method, and finally reading the output into VOTCA\_CT.

\subsubsection{Generation of directory structure and input coordinates}
First, hopping sites and a neighbor list need to be generated from the atomistic topology and trajectory. Rigid fragments are resubstituted into the molecular geometries and the neighbor list is defined based on a cutoff distance between such fragments. This can be achieved by running, e.g.  
\begin{verbatim}
ctp_map [required options] --db state.db --nframes 1 --first-frame 1
\end{verbatim}
to generate a state file is for first frame of the trajectory. Then a neighbor list can be determined from the cutoff defined in {\tt main.xml} (see \ref{list:neighborcut_xml}).
\lstinputlisting[
  language=XML,
  label=list:neighborcut_xml,
  stringstyle=\ttfamily\footnotesize,
  showstringspaces=false,
  caption={\small Definition of neighborlist cutoff in {\tt main.xml}}.] {./programs/neighborcut.xml}
by calling
\begin{verbatim}
ctp_run --opt main.xml --s segments.xml --db state.db --exec neighborlist
\end{verbatim}
and from that the full file and directory structure needed by {\tt ctp\_dipro} is written by 
\begin{verbatim}
ctp_run --opt main.xml --s segments.xml --db state.db --exec pairdump
\end{verbatim}
This requires the specification of the content of listing \ref{list:pairdump_xml} in {\tt main.xml}.
\lstinputlisting[
  language=XML,
  label=list:pairdump_xml,
  stringstyle=\ttfamily\footnotesize,
  showstringspaces=false,
  caption={\small Block of pairdump options required in {\tt main.xml} for generating the directory structure and coordinate files for {\tt ctp\_dipro}.}] {./programs/pairdump.xml}
After this, the following directories and files have been created:
\begin{verbatim}
|----frame1/
| |----mol_1/
| | |----mol_1.xyz
| | [...]
| |----pair_1_2/
| | |----dim/
| | | |----pair_1_2.xyz
| | [...]
\end{verbatim}

\subsubsection{Calculating the transfer integrals}
Before starting the quantum-chemical calculations with either {\tt Gaussian} or {\tt Turbomole}, make sure that the respective environments for these programs are set. 

First, for each molecule a converged electronic structure calculatiom has to be performed by running 
\begin{verbatim}
ctp_dipro --monomer QCP [METHOD]

QCP:   G for Gaussian09
       T for Turbomole

METHOD: func/basis (optional)
        overrides default functional/basisset combination
        defaults: pbepbe/6-311G** Gaussian09
                  pbe/def-TZVP    Turbomole
\end{verbatim}
in each {\tt mol\_*} directory. If no method is specified, {\tt ctp\_dipro} defaults to running a DFT calculation with the PBE functional and a 6-311G** basis set in {\tt Gaussian} and def-TZVP in {\tt Turbomole}. Note that {\tt OpenBabel} needs to be installed is {\tt Turbomole} is used. It is recommended to perform these calculations in batch mode on some kind of cluster system. Since it can eventually happen that files are not written back correctly, one should check if all files that are needed for the pair runs are present by executing in the {\tt frame*} directory
\begin{verbatim}
ctp_dipro --check N M QCP

N:   First monomer to test
M:   Last monomer to test
QCP: G/T 
\end{verbatim}
A list of incomplete monomer calculations is written to file {\tt TROUBLE.mol}. If this is empty, one can proceed with running the pair calculations. For any directory {\tt pair\_A\_B}, the completed monomer calculations from the previous step have to be present in subdirectories {\tt mol\_A} and {\tt mol\_B}, e.g.
\begin{verbatim}
| | [...]
| |----pair_1_2/
| | |----molA/
| | |----molB/
| | |----dim/
| | | |----pair_1_2.xyz
| | [...]
\end{verbatim}
These subdirectories can be either copies or symbolic links, however, the most practical realization depends on the specifics of execute machine (e.g. local or network, hard disc etc.) so that these are not created automatically! Once these are created by the user, the transfer integral for the pair can be calculated by running {\tt ctp\_dipro --dimer} in the {\tt pair\_A\_B/dim} directory:
 \begin{verbatim}
ctp_dipro --dimer QCP [METHOD]

QCP:   G for Gaussian09
       T for Turbomole

METHOD: func/basis (optional)
        overrides default functional/basisset combination
        defaults: pbepbe/6-311G** Gaussian09
                  pbe/def-TZVP    Turbomole
\end{verbatim}
This command automatically generates a dimer input guess from the converged monomer orbitals, detects (pseudo-)degeneracies of HOMO or LUMO and calculates the required transfer integrals. As a result of the run, a file {\tt pair\_A\_B/TI.xml} as shown in listing \ref{list:TI_xml} is created (needs proper example).
\lstinputlisting[
  language=XML,
  label=list:TI_xml,
  stringstyle=\ttfamily\footnotesize,
  showstringspaces=false,
  caption={\small Example {\tt TI.xml} file created as the output of a DIPRO calculation. Due to slightly different implementations, the orbitals indices refer to monomer indices in a {\tt Gaussian} run but to indices in the merged dimer guess in a {\tt Turbomole} run.}] {./programs/TI.xml}

\subsubsection{Writing transfer integrals to state file}
After performing all transfer integral calculations, the resulting output files {\tt pair\_A\_B/TI.xml} have to be collected in the folder {\tt transfer\_integrals} as {\tt transfer\_integrals/pair\_A\_B.xml}. Then the transfer integrals are written into the state file using:
\begin{verbatim}
ctp_dipro --write TYPE ID FILE

TYPE:   e    - electrons
        h    - holes
        edeg - (pseudo-)degenerate electrons (Boltzmann-weighted)
        hdeg - (pseudo-)degenerate holes (Boltzmann-weighted)
ID:     Number of frame
FILE:   Name of state file 
\end{verbatim}
During the run, some sanity checks are performed, i.e., whether all transfer integrals are calculated, and whether a pair that is written exists in the state file. If there is an error, the tool aborts.


\subsection{Semi-empirical methods}
\label{sec:moo}

\newcommand{\moo}{MOO\xspace}
\index{electronic coupling!ZINDO}

An approximate method based on Zerner's Independent Neglect of Differential Overlap (ZINDO) has been described in Ref.~\cite{kirkpatrick_approximate_2008}. This semiempirical method is substantially faster than first-principles approaches, since it avoids the self-consistent calculations on each individual monomer and dimer. This allows to construct the matrix elements of the ZINDO Hamiltonian of the dimer from the weighted overlap of molecular orbitals of the two monomers. Together with the introduction of rigid segments, only a single self-consistent calculation on one isolated conjugated segment is required. All relevant molecular overlaps can then be constructed from the obtained molecular orbitals. This Molecular Orbital Overlap (MOO) method has been applied successfully to study charge transport, for instance, in discotic liquid crystals~\cite{kirkpatrick_columnar_2008,marcon_understanding_2009,feng_towards_2009},
polymers~\cite{ruehle_multiscale_2010}, or partially disordered organic crystals~\cite{vehoff_charge_2010-1,vehoff_charge_2010-2,vehoff_charge_2010}.

The main advantage of the molecular orbital overlap \moo library is {\em fast} evaluation of electronic coupling elements. A detailed description of the method is provided in ref.~\cite{kirkpatrick_approximate_2008}. Please site this paper if you are using the method. Note that \moo is based on the semi-empirical ZINDO Hamiltonian and therefore has limited applicability. The general advice is to first compare the accuracy of the \moo method to the DFT-based calculations. 

\moo can be used both in a sandalone mode and as a \calculator of the \votcactp. \moo constructs the Fock operator of a dimer from the  molecular orbitals of monomers by translating and rotating the orbitals and therefore requires the optimized geometry of the molecule and the projection coefficients of the molecular on atomic orbitals. 


\subsubsection{Standalone mode}
For a standalone mode program \overlap is provided 
\begin{verbatim}
 moo_overlap --conjseg benzene.xml --pos1 benzene1.pos --pos2 benzene2.pos
\end{verbatim}
Its input requires a description of two conjugated segments (\texttt{benzene.xml}, positions and orientations of the molecules and the files with molecular coordinates and orbitals. The structure of the files is shown in listings \ref{list:benzene_xml} and  \ref{list:benzene_pos}.
\vskip 0.1cm
\lstinputlisting[
  language=XML,
  label=list:benzene_xml,
  stringstyle=\ttfamily\footnotesize,
  showstringspaces=false,
  caption={\small \texttt{benzene.xml} file with the description of the benzene molecule, which is also a single conjugated segment and a rigid fragment.}] {./programs/benzene.xml}

\vskip 0.1cm

\lstinputlisting[
  language=XML,
  label=list:benzene_pos, 
  stringstyle=\ttfamily\footnotesize,
  showstringspaces=false,
  caption={\small \texttt{benzene1.pos} file which describes the position and orientation of the molecule. The name of the molecule is followed by three coordinates (relative to the center of mass of the supplied \texttt{xyz} file and then by nine elements of the rotation matrix $a_{ij} = e_i e^\text{mol}_j $. The reference coordinate frame is determined from the provided \texttt{xyz} file.}] {./fig/moo/moo_overlap/benzene1.pos}


\subsubsection{Calculator of \votcactp}
Semi-empirical method of evaluation of electronic couplings in a morphology is provided by the \integrals \calculator. In addition to definitions of conjugated segments, atomistic trajectory, and state file, the program needs coordinates and orbitals of conjugated segments. Coordinates are stored in \xyz files with four columns, first being the atom type and the next three atom coordinates. This is a standard \texttt{xyz} format without a header. Note that the atom order in \xyz files can be different from that of the mapping files. The correspondence between the two is established in the file which defines conjugated segments.

