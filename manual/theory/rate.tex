\section{Charge transfer rate}
\label{sec:rates}

Charge transfer rates\index{charge transfer rate} can be postulated based on intuitive physical considerations, as it is done in the Gaussian disorder models~\cite{walker_electrical_2002,baessler_charge_1993,borsenberger_charge_1991,pasveer_unified_2005}. Alternatively, charge transfer theories can be used to evaluate rates from quantum chemical calculations~\cite{bredas_molecular_2009,coropceanu_charge_2007,bredas_charge-transfer_2004,nelson_modeling_2009,baumeier_density-functional_2010,ruhle_microscopic_2011}. In spite of being significantly more computationally demanding, the latter approach allows to link the chemical and electronic structure, as well as the morphology, to charge dynamics.

\subsection{Classical charge transfer rate}
\label{sec:rate_classical}
\index{charge transfer rate!classical}

The high temperature limit of classical charge transfer  theory~\cite{marcus_electron_1993,hutchison_hopping_2005} is often used as a trade-off between theoretical rigor and computational complexity. It captures key parameters which influence charge transport while at the same time providing an analytical expression for the rate. Within this limit, the transfer rate for a charge to hop from a site $i$ to a site $j$ reads
%
\begin{equation}
\omega_{ij}  = \frac{2 \pi}{\hbar}  \frac{ J_{ij}^2 }{\sqrt{ 4 \pi \lambda_{ij} k_\text{B}T}} \exp \left[
-\frac{\left(\Delta E_{ij}-\lambda_{ij}\right)^2}{4 \lambda_{ij}
k_\text{B} T} \right],
\label{equ:marcus}
\end{equation}
%
where $T$ is the temperature, $\lambda_{ij} = \lambda_{ij}^\text{int} + \lambda_{ij}^\text{out}$ is the \slink{sec:reorganization}{reorganization energy}, which is a sum of intra- and inter-molecular (outersphere) contributions, $\Delta E_{ij}$ is the \slink{sec:site_energies}{site-energy difference}, or driving force, and $J_{ij}$ is the \slink{sec:transfer_integrals}{electronic coupling element}, or transfer integral. 


\subsection{Semi-classical bimolecular rate}
\label{sec:rate_bimolecular}
\index{charge transfer rate!bimolecular}

\newcommand{\indM}{l}
\newcommand{\indN}{m}
\newcommand{\lb}[1]{\langle #1 |}
\newcommand{\rb}[1]{| #1 \rangle}
\newcommand{\rbt}[1]{ #1 \rangle}

The main assumptions in eq.~(\ref{equ:marcus}) are non-adiabaticity (small electronic coupling and  charge transfer between two diabatic, non-interacting states), and harmonic promoting modes, which are treated classically. At ambient conditions, however, the intramolecular promoting mode, which roughly corresponds to C-C bond stretching, has a vibrational energy of $\hbar\omega \approx 0.2\, \unit{eV} \gg k_\text{B}T$ and should be treated quantum-mechanically. The outer-sphere (slow) mode has much lower vibrational energy than the intramolecular promoting mode, and therefore can be treated classically. The weak interaction between molecules also implies that each molecule has its own, practically independent, set of quantum mechanical degrees of freedom.


A more general, quantum-classical expression for a bimolecular multi-channel rate is derived in the Supporting Information of ref.~\cite{ruhle_microscopic_2011} and has the following form

\begin{align}
 \omega_{ij}= \frac{2\pi}{\hbar}  \frac{|J_{ij}|^2}{\sqrt{4\pi \lambda_{ij}^\text{out} k_\text{B}T}}
 \sum_{\indM',\indN'=0}^\infty
 |\lb{\chi_{i0}^c}\rbt{\chi_{i\indM'}^n}|^2 |\lb{\chi_{j0}^n}\rbt{\chi_{j\indN'}^c}|^2
%\nonumber \\&&
\exp
\left\{ -\frac{ \left[ \Delta E_{ij}-\hbar(\indM'\omega_i^n+\indN'\omega_j^c) -\lambda_{ij}^\text{out} \right]^2}{4\lambda_{ij}^\text{out} k_\text{B}T}
\right\} .
\label{equ:jjortner}
\end{align}
If the curvatures of intramolecular PES of charged and neutral states of a molecule are different, that is $\omega_i^c\neq\omega_i^n$, the corresponding reorganization energies, $\lambda_i^{cn}=\frac{1}{2}[\omega_i^n(q_i^n-q_i^c)]^2$ and $\lambda_i^{nc}=\frac{1}{2}[\omega_i^c(q_i^n-q_i^c)]^2$, will also differ. In this case the Franck-Condon (FC) factors for discharging of molecule $i$ read \cite{chang_new_2005}
\begin{align}
%\begin{split}
%&
|\lb{\chi_{i0}^c}\rbt{\chi_{i\indM'}^n}|^2 =
\frac{2}{2^{l'}l'!} \frac{\sqrt{\omega_i^c\omega_i^n}}{(\omega_i^c+\omega_i^n)} \exp\left( -|s_i| \right)
%\nonumber \\
%& \times
 \left[ \sum_{\substack{k=0\\k\,\text{even}}}^{\indM'} {\indM' \choose k}
\left( \frac{2 \omega_i^c }{\omega_i^c+\omega_i^n}\right)^{k/2} \frac{k!}{(k/2)!}
H_{\indM'-k} \left( \frac{s_{i}}{\sqrt{2S^{cn}_i}}\right)
\right]^2
\, ,
%\end{split}
\end{align}
where $H_n(x)$ is a Hermite polynomial, $s_i=2\sqrt{\lambda_i^{nc}\lambda_i^{cn}} / \hbar(\omega_i^c+\omega_i^n)$, and $S^{cn}_i=\lambda_i^{cn}/\hbar\omega_i^c$. The FC factors for charging of molecule $j$ can be obtained by substituting $(s_i,S^{cn}_i,\omega_i^c)$ with $(-s_j,S^{nc}_j, \omega_j^n)$. In order to evaluate the FC factors, the \slink{sec:eintramolecular}{internal reorganization energy} $\lambda_i^{cn}$ can be computed from the intramolecular PES.

\subsection{Semi-classical rate}
\label{sec:rate_semiclassical}
\index{charge transfer rate!semiclassical}

One can also use the quantum-classical rate with a common set of vibrational coordinates~\cite{may_charge_2011}
\begin{align}
 \omega_{ij} = \frac{2\pi}{\hbar}  \frac{|J_{ij}|^2}{\sqrt{4\pi \lambda_{ij}^\text{out} k_\text{B}T}}
 \sum_{N=0}^\infty \frac{1}{N!} \left( \frac{\lambda_{ij}^\text{int}}{\hbar\omega^\text{int}} \right)^{N}
  \exp \left( - \frac{\lambda_{ij}^\text{int}}{\hbar\omega^\text{int}}\right)
%\nonumber\\&&
\exp
\left\{ -\frac{ \left[ \Delta E_{ij}-\hbar N\omega^\text{int} -\lambda_{ij}^\text{out} \right]^2}{4\lambda_{ij}^\text{out} k_\text{B}T}
\right\} .
\label{equ:jortner}
\end{align}

Numerical estimates show that if  $\lambda_{ij}^\text{int} \approx \lambda_{ij}^\text{out}$ and $|\Delta E_{ij}| \ll \lambda_{ij}^\text{out}$ the rates are similar to those of ~\equ{marcus}. In general, there is no robust method to compute $\lambda_{ij}^\text{out}$~\cite{hoffman_reorganization_1996} and  both reorganization energies are often assumed to be of the same order of magnitude. In this case the second condition also holds, unless there are large differences in electron affinities or ionization potentials of neighboring molecules, e.g. in donor-acceptor blends.
\vskip 0.4cm
To calculate rates of the type specified in \xmloptions for all pairs in the \slink{sec:neighborlist}{neighbor list} and to save them into the \sqlstate file, run the \calc{rates} \calculator. Note that all required ingredients (\slink{sec:reorganization}{reorganization energies}, \slink{sec:transfer_integrals}{transfer integrals}, and \slink{sec:site_energies}{site energies} have to be calculated before).
\votcacommand{Calculation of transfer rates}{\cmdrates}
