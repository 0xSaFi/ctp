\chapter{Neighbor list}
\label{sec:neighborlist}

A list of neighboring conjugated segments, or neighbor list\index{neighbor list}, contains all pairs of conjugated segments for which \slink{sec:transfer_integrals}{coupling elements}, \slink{sec:reorganization}{reorganization energies}, \slink{sec:site_energies}{site energy differences}, and \slink{sec:rates}{rates} are evaluated.

Two segments are added to this list if the distance between centers of mass of any of their rigid fragments is below a certain cutoff. This allows neighbors to be selected on a criterion of minimum distance of approach rather than center of mass distance, which is useful for molecules with anisotropic shapes.

The neighbor list can be generated from the atomistic trajectory by using the \calc{neighborlist} \calculator. This calculator requires a cutoff, which can be specified in the \xmloptions file. The list is saved to the \sqlstate file:
\votcacommand{Generating a neighbor list}{\cmdnbl}

Sometimes it is convenient to filter the rigid fragments, for example if a conjugated segment has alkyl side chains and we do not want to add a pair of molecules which have conjugated cores far apart but side chains close to each other. In this case one can provide a list of ``active'' fragments. The search algorithm will only look at the pairs of segments, for which these fragments are located closer than the specified cutoff.

It can also happen that a completely custom neighbor list is required. An external list can be imported from a file which has the format pairID, segment1ID, segment2ID, segment1Type, segment2Type, e.g.,
\lstset{language=XML}
\begin{lstlisting}
1  1 2 DCV DCV  
2  1 3 DCV DCV     
3  2 3 DCV DCV 
\end{lstlisting}
