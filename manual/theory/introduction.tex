\chapter{Overview}
\label{sec:introduction}

Charge carrier and exciton dynamics in organic semiconductors can often be described as a sequence of charge/exciton transfer reactions between localized states. Transfer rates depend on \slink{sec:transfer_integrals}{electronic coupling elements}, \slink{sec:reorganization}{reorganization energies}, and \slink{sec:site_energies}{site energies}, which vary as a function of molecular positions and orientations. The purpose of the \votcactp package~ is to simplify the computational workflow for charge and exciton transport simulations, which is shown in \Fig{summary}. 

In this workflow, \slink{morphology}{atomistic morphology} is mapped onto \slink{segments}{conjugated segments}  and rigid fragments. If needed, rigid fragments are substituted with the quantum-mechanically optimized copies. The conjugated segments are then used to construct a \slink{neighborlist}{neighbor list}. For each pair of this list an \slink{transfer_integrals}{electronic coupling element}, a \slink{reorganization}{reorganization energy}, a \slink{site_energies}{driving force}, and eventually the \slink{rates}{rate} are evaluated.

Solid-state \slink{sec:site_energies}{ionization energies, electron affinities, and excited state energies} of conjugated segments are calculated perturbatively, as a sum of the gas-phase contribution and electrostatic and polarization interaction with the environment. Coulomb interactions can be evaluated using a cutoff or an aperiodic Ewald summation, available for both the bulk and slab geometries. These calculations require distributed atomic multipoles and polarizabilities for the neutral, cationic (IE), anionic (EA), or excited state.

Electronic \slink{sec:transfer_integrals}{coupling elements} between conjugated segments can be performed by \slink{sec:dipro}{projecting} the dimer orbitals on the respective diabatic states, approximated by the monomer orbitals. Interfaces to \gaussian and, to a lesser extent, to \turbomole are provided to perform these computationally demanding simulations. Alternatively, it is possible to use the fast \slink{sec:izindo}{molecular orbital overlap} method based on the semi-empirical INDO Hamiltonian. This method requires INDO molecular orbitals in the format provided by the \gaussian package.

The \slink{neighborlist}{neighbor list} and \slink{rates}{rates} define a directed graph. The corresponding master equation is solved using the \slink{kmc}{kinetic Monte Carlo} method, which allows to explicitly monitor the charge and exciton dynamics in the system as well as to calculate time- or ensemble averages of occupation probabilities, charge fluxes, correlation functions, and field-dependent mobilities. 

The package is organized in several \slink{sec:programs}{programs} executing \slink{sec:calculators}{calculators}. Results are stored in a \slink{statefile}{sql database} which is also used to restart simulations. In the following we describe individual steps required to perform charge and exciton transport simulations, the format of input and output files, and the complete reference of \slink{sec:programs}{programs} and \slink{sec:calculators}{calculators}, compiled from the installed code. A tutorial is available on the github \hyperref[https://github.com/votca/ctp-tutorials]{github.com/votca/ctp-tutorials}.
