\subsection{Extrapolation to nondispersive mobilities}
\label{sec:nondispersive}
Predictions of charge-carrier mobilities in partially disordered semiconductors  rely on charge transport simulations in systems which are only several nanometers thick. As a result, simulated charge transport might be dispersive for materials with large energetic disorder~\cite{scher_anomalous_1975,borsenberger_role_1993} and simulated mobilities are system-size dependent. In time-of-flight (TOF) experiments, however, a typical sample thickness is in the micrometer range and transport is often nondispersive. In order to link simulation and experiment, one needs to extract the nondispersive mobility from simulations of small systems, where charge transport is dispersive at room temperature.

Such extrapolation is possible if the temperature dependence of the nondispersive mobility is known in a wide temperature range. For example, one can use analytical results derived for one-dimensional models~\cite{derrida_velocity_1983,cordes_one-dimensional_2001,seki_electric_2001}. The mobility-temperature dependence can then be parametrized by simulating charge transport at elevated temperatures, for which transport is nondispersive even for small system sizes. This dependence can then be used to extrapolate to the nondispersive mobility at room temperature~\cite{lukyanov_extracting_2010}.

For \Alq, the charge carrier mobility of a periodic system of 512 molecules was shown to be more than three orders of magnitude higher than the nondispersive mobility of an infinitely large system~\cite{lukyanov_extracting_2010}. Furthermore, it was shown that the transition between the dispersive and nondispersive transport has a logarithmic dependence on the number of hopping sites $N$. Hence, a brute-force increase of the system size cannot resolve the problem for compounds with large energetic disorder $\sigma$, since $N$ increases exponentially with $\sigma^2$.
 

