\newcommand{\indM}{l} 
\newcommand{\indN}{m}
\newcommand{\lb}[1]{\langle #1 |}
\newcommand{\rb}[1]{| #1 \rangle}
\newcommand{\rbt}[1]{ #1 \rangle}

%\appendixpage
%\addappheadtotoc
\chapter{Bimolecular electron transfer rate}
\label{sec:rate_bimolecular}

\begin{figure*}[ht]
%    \includegraphics[width=1.0\textwidth]{fig/jortner_rate/marcus_parabolas_new}
   \caption{
(a) Potential energy surfaces of the charge transfer complex in a dimer representation. ET is from molecule $i$ to molecule $j$. In the initial state, $\rb{I_{00}}$, both molecules are in their vibrational ground states. In the final state, $\rb{F_{l'm'}}$, the neutral molecule $i$ is in vibrational state $l'$, while the charged molecule $j$ is in vibrational state $m'$. Initial and final states are coupled to a classical harmonic outer-sphere normal mode with mass weighted average coordinate $q$ and reorganization energy $\lambda_{ij}^\text{out}$. For small couplings $V_{I_{00}F_{l'm'}}$ the ET reaction takes place on the diabatic states (solid curves). 
%
(b) PES of molecule $i$ as a function of the averaged normal mode $q_i$. $l$ and $l'$ enumerate vibrational modes of the initial charged and the final neutral states. (c) Same as (b) for initially neutral molecule $j$. 
%
$\Delta U_i$ ($\Delta U_j$) is the internal energy difference while $\lambda_i^{cn}$ ($\lambda_j^{nc}$) is the  intramolecular reorganization energy for discharging molecule $i$ (charging molecule $j$). }
   \label{fig:marcus_parabolas}
\end{figure*}

In the case of a bimolecular electron transfer (ET) reaction the electron moves between two independent molecules. Therefore, one needs separate sets of coordinates for the donor and acceptor. Strictly speaking, the classical Marcus rate assumes a common set of vibrational coordinates and, as such, can not be used for bimolecular ET. Yet if the independent vibrational modes are harmonic, are treated classically, and the charging and discharging reorganization energies of the molecule are identical, one still obtains the Marcus-type ET rate with the intramolecular reorganization energy which is the sum of the reorganization energies of the donor and the acceptor~\cite{may_charge_2003}. Similarly, the classical treatment of the outer-sphere mode, which is due to rearrangement of the surrounding, allows to add its reorganization energy  to the intramolecular one.

However, the main issue with the classical Marcus rate is that the high-frequency intramolecular vibrational modes are energetically comparable to the C-C bond stretching mode. At room temperature $\hbar \omega_\text{CC} \sim 0.2\, \unit{eV} \gg k_\text{B}T \sim 0.025\, \unit{eV}$  and therefore these modes should be treated quantum mechanically. In fact, for a common set of intramolecular high-frequency  (quantum-mechanical) and an outer sphere low-frequency (classical) vibrational coordinates, a mixed quantum-classical multi-channel generalization of the Marcus formula is readily available~\cite{may_charge_2003}. Such generalization,  to the best of our knowledge, has not been made for the bimolecular ET rate, which requires independent sets of coordinates for donor and acceptor. The purpose of this section is to derive a quantum-classical expression for the ET rate with two independent, high-frequency vibrational modes and a common low-frequency outer sphere mode. 

Following Ref.~\cite{bredas_charge-transfer_2004} we assume that all intramolecular modes of a donor $i$ can be averaged into a mode with mass weighted coordinate ${q_i}$ and energy $\hbar\omega^{n}_i$ ($\hbar\omega^{c}_i$) for the molecule in a neutral (charged) state. Similar assumptions are made for the acceptor $j$. In addition, we allow for an averaged classical outer-sphere mode with mass weighted coordinate ${q}$ and energy $\hbar\omega^\text{out}_{ij}\ll k_\text{B}T$. This mode is common to both molecules and plays the role of the ET reaction coordinate~\cite{note_outer}.

In amorphous organic semiconductors the electronic coupling is usually small compared to both the energy of the classical vibrational mode and intermolecular reorganization energies. In this case the initial, $ \rb{I_{\indM\indN}}$, and final, $\rb{F_{\indM'\indN'}}$, states of the ET reaction are diabatic (non-interacting) dimer states which depend on the vibrational states (with quantum numbers $\indM, \indN, \indM, '\indN'$) of both molecules. The potential energy surfaces (PES) corresponding to these states are shown in \fig{marcus_parabolas}a. The PES for intramolecular degrees of freedom for molecules $i$ and $j$ are shown in \fig{marcus_parabolas}b and c, respectively.

For the contributions of the outer-sphere mode to initial and final states we introduce Hamiltonian functions
\begin{equation}
 H_{I,F}(q)=\frac{1}{2}\left[ \omega^\text{out} \left( q-q_{I,F} \right) \right]^2 \, ,
\end{equation}
where the equilibrium position in the initial (final) state $q_I$ ($q_F$) corresponds to the arrangement of all nuclear coordinates of molecules surrounding the ET complex when molecule $i$ ($j$) is charged.  The outer sphere reorganization energy, defined as $\lambda^\text{out}_{ij}=\frac{1}{2}\left[ \omega^\text{out}|q_I-q_F| \right]^2$,  is shown in~\fig{marcus_parabolas}a. It can be computed from the initial and final electric displacement fields of the charge-transfer complex. 

The complete Hamiltonian of the ET complex can now be written as
\begin{equation}
\begin{split}
 H_{ij}=&\hphantom{+}\sum_{\indM,\indN=0}^\infty \left( H_I(q) + E^{ij}_{\indM\indN} \right) \rb{I_{\indM\indN}} \lb{I_{\indM\indN}} %\\
        +\sum_{\indM',\indN'=0}^\infty \left( H_F(q) + E^{ji}_{\indN'\indM'} \right) \rb{F_{\indM'\indN'}} \lb{F_{\indM'\indN'}} %\\
        +\sum_{\indM,\indN,\indM',\indN'}V_{I_{\indM\indN}F_{\indM'\indN'}} \rb{I_{\indM\indN}} \lb{F_{\indM'\indN'}} +\text{h.c} \,, \\
 E^{ij}_{\indM\indN}=& \hphantom{+} U_i^{cC} + U_j^{nN} + E_i^\text{el} + E_i^\text{ext}
+ \hbar\left[\omega_i^c\left(\indM+\frac{1}{2}\right) + \omega_j^n\left(\indN+\frac{1}{2}\right) \right] \, , 
       \\
 E^{ji}_{\indN'\indM'}=&\hphantom{+} U_j^{cC} + U_i^{nN} + E_j^\text{el} + E_j^\text{ext} 
+ \hbar\left[\omega_j^c\left(\indN'+\frac{1}{2}\right) +\omega_i^n\left(\indM'+\frac{1}{2}\right)\right] \, .
\end{split}
\label{equ:hami}
\end{equation}
Here, a manifold of initial states, $\rb{I_{\indM \indN}} $, with quantum numbers $\indM$ ($\indN$) for intramolecular vibrations in molecule $i$ ($j$) and energy $E^{ij}_{\indM \indN}$, is coupled to a classical phonon bath $H_I(q)$. Transitions to the manifold of final states  $\rb{F_{\indM' \indN'}} $ where the charge has hopped from $i$ to $j$ are possible due to a coupling $V_{I_{\indM\indN}F_{\indM'\indN'}} $.
The initial, $E^{ij}_{\indM\indN}$,  and final, $E^{ji}_{\indN'\indM'}$, energies contain internal energies $U_i^{nN}$ and $U_i^{cC}$ ($U_j^{nN}$ and $U_j^{cC}$) of molecule $i$ (molecule $j$) in the neutral and charged ground states, the contributions of the external electric field, $E_i^\text{ext}$ and $E_j^\text{ext}$, and electrostatic interactions, $E_i^\text{el}$ and $E_j^\text{el}$, and respective oscillator energies.  

Within the Born-Oppenheimer approximation, a separation in terms of electronic and nuclear degrees of freedom gives
\begin{equation}
\begin{split}
 \rb{I_{\indM\indN}}&=\rb{\phi_i^c}\rb{\chi_{i\indM}^c}\rb{\phi_j^n}\rb{\chi^n_{j\indN}}\, ,\\
 \rb{F_{\indM'\indN'}}&=\rb{\phi^n_i}\rb{\chi^n_{i\indM'}}\rb{\phi_j^c}\rb{\chi_{j\indN'}^c}\, ,
\end{split}
\end{equation}
where $\phi_i^n$ ($\phi_i^c$) corresponds to the electronic part of the wave function, while $\chi_{i\indM}^n$  ($\chi_{i\indM}^c$) represents an $\indM$-th phonon mode of the neutral (charged) molecule $i$.

The coupling element $V_{I_{\indM\indN}F_{\indM'\indN'}}$ in~\equ{hami} can then be factorized in an electronic and nuclear parts 
\begin{equation}
V_{I_{\indM\indN}F_{\indM'\indN'}}=J_{ij} \lb{\chi_{i\indM}^c}\rbt{\chi_{i\indM'}^n} \lb{\chi_{j\indN}^n}\rbt{\chi_{j\indN'}^c}\,.
\end{equation}
Franck-Condon overlap integrals $\lb{\chi_{i\indM}^c}\rbt{\chi_{i\indM'}^n}$ ($\lb{\chi_{j\indN}^n}\rbt{\chi_{j\indN'}^c} $) describe couplings of vibrational modes $\indM,\indM'$ ($\indN,\indN'$) of the charged and neutral configurations of molecule $i$  ($j$). Exemplary modes are shown in~\fig{marcus_parabolas}b,c.

Since $k_\text{B}T\ll \hbar\omega_i^{c},\hbar\omega_j^{n}$ one can restrict the initial state to the vibrational ground-states $\indM=\indN=0$ while allowing tunneling to all vibrationally excited states $\indM'$ for molecule $i$ and $\indN'$ for molecule $j$. In other words, a single initial state $\rb{I_{00}}$ couples to a manifold of final states $\rb{F_{\indM',\indN'}}$. 
%
This assumes that ET is sufficiently slow compared to the relaxation of the intramolecular degrees of freedom, so that there is enough time for a complex to relax to its vibrational ground state between two consecutive ETs. 

The energy difference driving the reaction to channel ${\indM'\indN'}$  therefore is
\begin{equation*}
 \Delta E^{ij}_{\indM'\indN'}= E^{ij}_{00}-E^{ji}_{\indN'\indM'}=\Delta E_{ij} - \hbar (\omega_i^n\indM'+\omega_j^c\indN')\,,
\end{equation*}
where $\Delta E_{ij}=\Delta E_{ij}^\text{ext}+\Delta E_{ij}^\text{el}+\Delta E^\text{int}_{ij}$.

Assuming that $|V_{I_{00}F_{\indM'\indN'}}|\ll\lambda_{ij}^\text{out},\hbar\omega^\text{out}$ and using Fermi's golden rule with $V_{I_{00}F_{\indM'\indN'}}$ as a perturbation to the initial diabatic state, we obtain a multi-channel rate equation
\begin{equation}
%\begin{split}
\omega_{ij}=\sum_{\indM',\indN'=0}^\infty \frac{2\pi}{\hbar}|V_{I_{00}F_{\indM'\indN'}}|^2 
% \times  
\int dq f_I(q) \delta(\Delta E^{ij}_{\indM'\indN'} + H_I(q) -H_F(q))\, .
%\end{split}
\end{equation}
where the thermal averaging over the classical outer-sphere mode is performed by introducing a  canonical distribution function  $f_I(q)=Z^{-1}\exp(-H_I(q)/k_\text{B}T)$, with $Z=\int{dq \exp(-H_I(q)/k_\text{B}T)}$.

Energy conservation pins the transition to the crossing point of the diabatic PES (see~\fig{marcus_parabolas}a) resulting in 
\begin{eqnarray}
 \omega_{ij}= \frac{2\pi}{\hbar}  \frac{|J_{ij}|^2}{\sqrt{4\pi \lambda_{ij}^\text{out} k_\text{B}T}} 
 \sum_{\indM',\indN'=0}^\infty
 |\lb{\chi_{i0}^c}\rbt{\chi_{i\indM'}^n}|^2 |\lb{\chi_{j0}^n}\rbt{\chi_{j\indN'}^c}|^2 
%\nonumber \\&& 
\exp
\left\{ -\frac{ \left[ \Delta E_{ij}-\hbar(\indM'\omega_i^n+\indN'\omega_j^c) -\lambda_{ij}^\text{out} \right]^2}{4\lambda_{ij}^\text{out} k_\text{B}T}
\right\} .
\label{equ:jjortner}
\end{eqnarray}
\Equ{jjortner} is the quantum-classical expression for the bimolecular ET rate with two independent, high-frequency vibrational modes and one classical common outer-sphere mode. It is the main result of this section. 

If the curvatures of intramolecular PES of charged and neutral states of a molecule are different, that is $\omega_i^c\neq\omega_i^n$, the corresponding reorganization energies, $\lambda_i^{cn}=\frac{1}{2}[\omega_i^n(q_i^n-q_i^c)]^2$ and $\lambda_i^{nc}=\frac{1}{2}[\omega_i^c(q_i^n-q_i^c)]^2$, will also differ. In this case the Franck-Condon (FC) factors for discharging of molecule $i$ read \cite{chang_new_2005}
\begin{equation}
%\begin{split}
%&
|\lb{\chi_{i0}^c}\rbt{\chi_{i\indM'}^n}|^2 = 
\frac{2}{2^{l'}l'!} \frac{\sqrt{\omega_i^c\omega_i^n}}{(\omega_i^c+\omega_i^n)} \exp\left( -|s_i| \right)
%\nonumber \\
%& \times
 \left[ \sum_{\substack{k=0\\k\,\text{even}}}^{\indM'} {\indM' \choose k} 
\left( \frac{2 \omega_i^c }{\omega_i^c+\omega_i^n}\right)^{k/2} \frac{k!}{(k/2)!}
H_{\indM'-k} \left( \frac{s_{i}}{\sqrt{2S^{cn}_i}}\right) 
\right]^2
\, ,
%\end{split}
\end{equation}
where $H_n(x)$ is a Hermite polynomial, $s_i=\frac{2\sqrt{\lambda_i^{nc}\lambda_i^{cn}}}{\hbar(\omega_i^c+\omega_i^n)}$, and $S^{cn}_i=\lambda_i^{cn}/\hbar\omega_i^c$. The FC factors for charging of molecule $j$ can be obtained by substituting $(s_i,S^{cn}_i,\omega_i^c)$ with $(-s_j,S^{nc}_j, \omega_j^n)$. In order to evaluate the FC factors, the internal reorganization energy $\lambda_i^{cn}$ can be computed from the intramolecular PES, as shown in~\fig{marcus_parabolas}b,c. 

To conclude the section, we compare the bimolecular quantum-classical rate, \equ{jjortner}, the classical bimolecular Marcus rate, eq.~(1) of the main text, and the quantum-classical Jortner rate with a common set of vibrational coordinates~\cite{may_charge_2003}
\begin{eqnarray}
 \omega_{ij} = \frac{2\pi}{\hbar}  \frac{|J_{ij}|^2}{\sqrt{4\pi \lambda_{ij}^\text{out} k_\text{B}T}} 
 \sum_{N=0}^\infty \frac{1}{N!} \left( \frac{\lambda_{ij}^\text{int}}{\hbar\omega^\text{int}} \right)^{N} 
  \exp \left( - \frac{\lambda_{ij}^\text{int}}{\hbar\omega^\text{int}}\right) 
%\nonumber\\&& 
\exp
\left\{ -\frac{ \left[ \Delta E_{ij}-\hbar N\omega^\text{int} -\lambda_{ij}^\text{out} \right]^2}{4\lambda_{ij}^\text{out} k_\text{B}T}
\right\} .
\label{equ:jortner}
\end{eqnarray}

If $\omega_i^c=\omega_i^n=\omega_i$ ($\lambda_i^{nc}=\lambda_i^{cn}=\lambda_i$), the Franck-Condon factor simplifies to
\begin{equation}
|\lb{\chi_{i0}}\rbt{\chi_{i\indM'}}|^2 = \frac{1}{\indM'!} \left( \frac{\lambda_i}{\hbar\omega_i} \right)^{\indM'} \exp \left( - \frac{\lambda_i}{\hbar\omega_i}\right)\,.
\end{equation}
If this simplification is applicable for both donor and acceptor molecules,  \equ{jjortner} becomes identical to the quantum-classical rate~\equ{jortner} with $\lambda_{ij}^\text{int}=\lambda_i+\lambda_j$.

\begin{figure*}[ht]
%   \includegraphics[width=\linewidth]{fig/jortner_rate/compare_rates}
   \caption{ (a) Scaled hopping rates, $\bar{\omega}_{ij} = \omega_{ij}  J_{ij}^{-2} (2\pi)^{-1} \hbar\sqrt{4\pi k_\text{B}T}$, calculated using the classical Marcus, Jortner quantum mechanical~\equ{jortner} and bimolecular multichannel~\equ{jjortner} rate expressions. 
%
Outer sphere reorganization energy $\lambda_{ij}^\text{out}=0.05\, \unit{eV}$,  $\lambda^{cn}_i=\lambda_j^{cn}=0.14\, \unit{eV}$, and $\lambda^{nc}_i=\lambda^{nc}_j=0.09\, \unit{eV}$ (all added in the classical Marcus rate while the latter two are added for the Jortner rate). 
%
Intramolecular vibrations have averaged frequency $\hbar\omega_i^\text{int}=\hbar\omega_j^\text{int}=0.2\, \unit{eV}$ for the Jortner rate while $\hbar\omega_i^n=0.2\, \unit{eV}$ and $\hbar\omega_j^c=\hbar\omega_i^n \sqrt{\lambda_j^{nc}}/\sqrt{\lambda_j^{cn}}$ for the bimolecular rate. 
%
(b) The same but for $\lambda_{ij}^\text{out}=0.1\, \unit{eV}$. 
%
(c) Histogram of rates at a field of $10^8\, \unit{Vm^{-1}}$ for the Marcus and Jortner rates with distance dependent $\lambda_{ij}^\text{out}<0.08\, \unit{eV}$ for the neighborlist pairs and constant $\lambda^\text{int}=0.23\, \unit{eV}$. A small difference can be seen in the tail of small rates.}
   \label{fig:mj_comparison}
\end{figure*}

% discussion of the equation
To compare the quantum-mechanical and classical rates, intramolecular hole reorganization energies of \Alq, $\lambda^{cn}_i=0.14\,\unit{eV}$ and $\lambda^{nc}_i=0.09\,\unit{eV}$ were used. We also assumed that $\omega_i^n=0.2\,\unit{eV}$ and $\omega_i^c=\omega_i^n\sqrt{\lambda^{nc}_i / \lambda^{cn}_i}$. Due to the uncertainty in determining $\lambda_{ij}^\text{out}$, two cases are considered, $\lambda_{ij}^\text{out}=0.05\,\unit{eV}$ (\fig{mj_comparison}a) and $\lambda_{ij}^\text{out}=0.10\,\unit{eV}$ (\fig{mj_comparison}b). Note that the estimate made in the main text predicts $\lambda_{ij}^\text{out} < 0.08\,\text{eV}$. In both cases we used fixed, molecular-separation independent, $\lambda_{ij}^\text{out}$. 

\Fig{mj_comparison} shows that the main difference between the quantum-classical and classical rates is the tail of smaller rates for large negative $\Delta E$ (endothermal hopping) and higher rates for large positive $\Delta E$ (exothermal hopping).  \Fig{mj_comparison}c also shows the corresponding distributions of rates for all pairs from the neighbor list for 512 molecules of  amorphous \Alq. Here we used the distance-dependent $\lambda_{ij}^\text{out}$ from the neighbor list as computed from dielectric displacement fields with the Pekar factor of $c_p=0.01$. One can see that the distributions are practically on top of each other (except for very small rates) and hence will lead to similar charge dynamics. 

In general, our observation is that for a situation with (i) intramolecular reorganization energy similar to the outer sphere one ($\lambda_{ij}^\text{int} \sim \lambda_{ij}^\text{out}$), (ii) driving force $\Delta E_{ij}$ small compared to the intramolecular reorganization energy, and (iii) $\lambda \sim \hbar \omega$, the classical (eq.~(1) of the main text) and semi-classical  (\equ{jortner} and \equ{jjortner}) expressions lead to quantitatively similar rates.
However, for systems with large $\Delta E_{ij}$, such as donor-acceptor mixtures, \equ{jjortner} or \equ{jortner} should be used. In this case a rather accurate estimate of the outer sphere reorganization energy is required~\cite{mcmahon_evaluation_2010}.

