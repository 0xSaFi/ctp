
\chapter{\moo standalone mode}
\label{sec:moo_standalone}

For a standalone mode program \overlap is provided 
\begin{verbatim}
 moo_overlap --conjseg benzene.xml --pos1 benzene1.pos --pos2 benzene2.pos
\end{verbatim}
Its input requires a description of two conjugated segments (\texttt{benzene.xml}, positions and orientations of the molecules and the files with molecular coordinates and orbitals. The structure of the files is shown in listings \ref{list:benzene_xml} and  \ref{list:benzene_pos}.
\vskip 0.1cm
\lstinputlisting[
  language=XML,
  label=list:benzene_xml,
  stringstyle=\ttfamily\footnotesize,
  showstringspaces=false,
  caption={\small \texttt{benzene.xml} file with the description of the benzene molecule, which is also a single conjugated segment and a rigid fragment.}] {./programs/benzene.xml}

\vskip 0.1cm

\lstinputlisting[
  language=XML,
  label=list:benzene_pos, 
  stringstyle=\ttfamily\footnotesize,
  showstringspaces=false,
  caption={\small \texttt{benzene1.pos} file which describes the position and orientation of the molecule. The name of the molecule is followed by three coordinates (relative to the center of mass of the supplied \texttt{xyz} file and then by nine elements of the rotation matrix $a_{ij} = e_i e^\text{mol}_j $. The reference coordinate frame is determined from the provided \texttt{xyz} file.}] {./fig/moo/moo_overlap/benzene1.pos}

