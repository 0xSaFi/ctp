\chapter{Reference}

\section{Programs}
\label{ref:programs}
Programs execute specific tasks (calculators). 

\input{reference/programs/all}

\section{Calculators}
\label{ref:calculators}
\label{sec:calculators}

Calculator is a piece of code which computes specific system properties, such as site energies, transfer integrals, etc. \ctprun, \kmcrun are wrapper programs which executes such calculators. The generic syntax is 
\vskip 0.2cm
{\noindent \small \ctprun \exe \texttt{"calc1, calc2, ..."} \opt \xmloptions }
\vskip 0.2cm
%
File \xmloptions lists all options needed to run a specific calculator. The format of this file is explained in listing~\ref{list:calc}. A complete list of calculators is given in the \refcalc reference section.
%
\lstinputlisting[label=list:calc, 
 caption={\small A part of the \xmloptions file with options for the \texttt{calculator\_name\{1,2\}} \refcalc.
}]{./reference/calculators.xml}

A list of all calculators and their short descriptions can be obtain using 
\vskip 0.1cm
{\noindent \small \ctprun \texttt{-{}-list} }
\vskip 0.1cm

A detailed description of all options of a specific calculator(s) is available via
\vskip 0.1cm
{\noindent \small \ctprun \texttt{-{}-desc calc1,calc2,...} }

\subsection{Calculators}
\label{sec:calculators}

Calculator is a piece of code which computes specific system properties, such as site energies, transfer integrals, etc. \ctprun is a wrapper program which executes all calculators. The generic syntax is 
\begin{verbatim}
  ctp_run --exec "calc1, calc2, ..." --opt options.xml
\end{verbatim}
%
File \texttt{options.xml} lists all options needed to run a specific calculator. The format of this file is explained in listing~\ref{list:calc}. A complete list of calculators is given in the \refcalc reference section.
%
\lstinputlisting[label=list:calc, 
 caption={\small A part of the \texttt{options.xml} file with options for the \texttt{calculator\_name\{1,2\}} \refcalc.
}]{./programs/calculators.xml}

\vfill

\section{Common options}
\label{ref:options}
%\setdefaultleftmargin{0.8em}{0.8em}{0.8em}{0.8em}{0.8em}{0.8em}
\rowcolors{1}{invisiblegray}{white}
{\small 
\input{reference/xml/options.xml}
}
\vfill
