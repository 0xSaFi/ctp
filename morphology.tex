\section{Material morphology}
\label{sec:morphology}

There is no generic recipe on how to predict a large-scale atomistically-resolved morphology of an organic semiconductor. The required methods are system-specific: for ultra-pure crystals, for example, density-functional methods can be used provided the crystal structure is known from experiment. For partially disordered organic semiconductors, however, system sizes much larger than a unit cell  are required. Classical molecular dynamics or Monte Carlo techniques are then the methods of choice. 

In molecular dynamics, atoms are represented by point masses which interact via empirical potentials prescribed by a force-field. Force-fields are parametrized for a limited set of compounds and their refinement is often required for new molecules. In particular, special attention shall be paid to torsion potentials between successive repeat units of conjugated polymers or between functional groups and the $\pi$-conjugated system. First-principles methods can be used to characterize the missing terms of the potential energy function. 

Self-assembling materials, such as soluble oligomers, discotic liquid crystals, block copolymers, partially crystalline polymers, etc., are the most complicated to study. The morphology of such systems often has several characteristic length scales and can be kinetically arrested in a thermodynamically non-equilibrium state. For such systems, the time- and length-scales of atomistic simulations might be insufficient to equilibrate or sample desired morphologies. In this case, systematic coarse-graining can be used to enhance sampling~\cite{ruehle_versatile_2009}. Note that the coarse-grained representation must reflect the structure of the atomistic system and allow for back-mapping to the atomistic resolution.

Here we assume that the morphology is already known, that is we know how the topology and the coordinates of all atoms in the systems at a given time. \votcact can read standard \gromacs topology files. Custom definitions of \hyperref[sec:atomistic]{atomistic topology} via \xml files are also possible. Since the description of the atomistic topology is the first step in the charge transport simulations, it is important to follow simple conventions on how the system is partitioned on molecules, residues, and how atoms are named in the topology. Required input files are described in section \hyperref[sec:atomistic]{Atomistic topology}. 
