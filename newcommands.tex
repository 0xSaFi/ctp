\input{hgid}

\newcommand{\equ}[1]{eq.~\eqref{equ:#1}}
\newcommand{\Equ}[1]{Eq.~\eqref{equ:#1}}
\newcommand{\fig}[1]{figure~\ref{fig:#1}}
\newcommand{\Fig}[1]{Figure~\ref{fig:#1}}
\newcommand{\sect}[1]{section~\ref{sec:#1}}

\newcommand{\slink}[2]{\hyperref[#1]{#2}}


\newcommand{\xml}{XML\xspace}
\newcommand{\gromacs}{GROMACS\xspace}
\newcommand{\gaussian}{GAUSSIAN\xspace}
\newcommand{\turbomole}{TURBOMOLE\xspace}
\newcommand{\tinker}{TINKER\xspace}
\newcommand{\dipro}{DIPRO\xspace}

\newcommand{\Alq}{$\mathrm{Alq}_3$\xspace}
\newcommand{\dcvt}{DCV2T\xspace}

\newcommand{\xyz}{\texttt{geometry.xyz}\xspace}
\newcommand{\orb}{\texttt{zindo.orb}\xspace}
\newcommand{\votcactp}{{\MakeUppercase{votca-ctp}}\xspace}

\newcommand{\calculator}{\hyperref[sec:calculators]{calculator}\xspace}

\newcommand{\xmloptions}{\texttt{options.xml}\xspace}
\newcommand{\xmlcsg}{\hyperref[sec:xmlmap]{\texttt{map.xml}}\xspace}
\newcommand{\xmlsegments}{\hyperref[sec:xmlsegments]{\texttt{segments.xml}}\xspace}
\newcommand{\sqlstate}{\hyperref[sec:statefile]{\texttt{state.db}}\xspace}
\newcommand{\topology}{\texttt{topol.tpr}\xspace}
\newcommand{\trajectory}{\texttt{traj.xtc}\xspace}

\newcommand{\opt}{\texttt{{ -}o}\xspace}
\newcommand{\seg}{\texttt{{ -}s}\xspace}
\newcommand{\sql}{\texttt{{ -}f}\xspace}
\newcommand{\exe}{\texttt{{ -}e}\xspace}
\newcommand{\tpl}{\texttt{{ -}t}\xspace}
\newcommand{\csg}{\texttt{{ -}m}\xspace}
\newcommand{\trj}{\texttt{{ -}c}\xspace}


\newcommand{\refcalc}{\hyperref[ref:calculators]{calculators}\xspace}

\newcommand{\overlap}{\hyperref[prog:moo_overlap]{\texttt{moo\_overlap}}\xspace}
\newcommand{\ctprun}{\hyperref[prog:ctp_run]{\texttt{ctp\_run}}\xspace}
\newcommand{\ctpmap}{\hyperref[prog:ctp_map]{\texttt{ctp\_map}}\xspace}
\newcommand{\ctpdipro}{\hyperref[prog:ctp_dipro]{\texttt{ctp\_dipro}}\xspace}

\newcommand{\sqlite}{\texttt{sqlite3}\xspace}
\newcommand{\sqlconjsegproperties}{\texttt{conjseg\_properties}\xspace}
\newcommand{\sqlconjsegs}{\texttt{conjsegs}\xspace}
\newcommand{\sqlmolecules}{\texttt{molecules}\xspace}
\newcommand{\sqlpairintegrals}{\texttt{pairintegrals}\xspace}
\newcommand{\sqlpairproperties}{\texttt{pairproperties}\xspace}
\newcommand{\sqlpairs}{\texttt{pairs}\xspace}
\newcommand{\sqlrigidfragproperties}{\texttt{rigidfrag\_properties}\xspace}
\newcommand{\sqlrigidfrags}{\texttt{rigidfrags}\xspace}
\newcommand{\sqlframes}{\texttt{frames}\xspace}


\newcommand{\suggestion}[1]{{\color{red}SUGGESTION: #1}}

\newcommand{\segmentref}[1]{segments.#1}
\newcommand{\segmentopt}[1]{\hyperlink{\segmentref{#1}}{\StrSubstitute{#1}{_}{\_}}\xspace}
\newcommand{\calcref}[1]{#1}
\newcommand{\calcopt}[1]{\hyperlink{\calcref{#1}}{\StrSubstitute{#1}{_}{\_}}\xspace}

\newcommand{\calc}[1]{\hyperref[calc:#1]{\texttt{#1}}\xspace}

\def\bibsection{%
    \chapter*{Bibliography}%
    \addcontentsline{toc}{chapter}{Bibliography}
}

\renewcommand*{\showkeyslabelformat}[1]{{\normalfont\tiny\sffamily#1}}
\definecolor{refkey}{rgb}{1,0,0}
\definecolor{labelkey}{rgb}{1,0,0}
