\chapter{Introduction}
\label{sec:introduction}

Charge carrier dynamics in an organic semiconductor can often be described in terms of charge hopping between localized states. The hopping rates depend on \slink{transfer_integrals}{electronic coupling elements}, \slink{reorganization}{reorganization energies}, and \slink{site_energies}{driving forces}, which vary as a function of position and orientation of the molecules.  The exact evaluation of these contributions in a molecular assembly is computationally prohibitive. Various, often semi-empirical, approximations are employed instead. The purpose of the \votcactp package~\cite{ruehle_microscopic_2011} is to simplify the workflow for charge transport simulations, provide a uniform error-control for the methods, flexible platform for their development, and eventually allow in silico prescreening of organic semiconductors for specific applications. 

The toolkit is implemented using modular concepts introduced earlier in the Versatile Object-oriented Toolkit for Coarse-graining Applications (VOTCA)~\cite{ruehle_versatile_2009}. The VOTCA structures are adapted to reading atomistic trajectories, mapping them onto \slink{conjugated_segments}{conjugated segments and rigid fragments}, and substituting (if needed) rigid fragments with the optimized copies. 

The \slink{izindo}{molecular orbital overlap} module calculates electronic coupling elements between  conjugated segments from the corresponding molecular orbitals. It relies on the semi-empirical INDO Hamiltonian and molecular orbitals in the format provided by the \gaussian package. An alternative,  \slink{dft}{density-functional} based approach, has interfaces to the \gaussian and \turbomole packages. An interface to the \tinker package is provided for calculations of electrostatic and polarization contributions to \slink{site_energies}{energetic disorder}. 

The  \slink{kmc}{kinetic Monte Carlo} module reads in the \slink{neighborlist}{neighbor list}, \slink{morphology}{site coordinates}, and \slink{rates}{hopping rates} and performs charge dynamics simulations using either periodic boundary conditions or charge sources and sinks. 

The toolkit is written as a combination of modular C++ code and scripts. The data transfer between programs is implemented via a \slink{statefile}{state file} (sql database), which is also used to restart simulations. Analysis functions and most of the calculation routines are encapsulated by using the observer pattern~\cite{gamma_design_1995} which allows the implementation of new functions as individual modules.