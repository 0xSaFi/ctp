\section{Charge transfer rate}

Charge transfer rates can be postulated based on intuitive physical considerations, as it is done in the Gaussian disorder models~\cite{walker_electrical_2002,baessler_charge_1993,borsenberger_charge_1991,pasveer_unified_2005}. Alternatively, charge transfer theories can be used to evaluate rates from quantum chemical calculations~\cite{bredas_molecular_2009,coropceanu_charge_2007,bredas_charge-transfer_2004,nelson_modeling_2009,baumeier_density-functional_2010,kirkpatrick_approximate_2008}. In spite of being significantly more computationally demanding, the latter approach allows to link the chemical and electronic structure, as well as the morphology, to charge dynamics.

\subsection{Classical charge transfer rate}
The high temperature limit of classical charge transfer  theory~\cite{marcus_electron_1993,hutchison_hopping_2005} is often used as a trade-off between theoretical rigor and computational complexity. It captures key parameters which influence charge transport while at the same time providing an analytical expression for the rate. Within this limit, the transfer rate for a charge to hop from a site $i$ to a site $j$ reads
%
\begin{equation}
\omega_{ij}  = \frac{2 \pi}{\hbar}  \frac{ J_{ij}^2 }{\sqrt{ 4 \pi \lambda_{ij} k_\text{B}T}} \exp \left[
-\frac{\left(\Delta E_{ij}-\lambda_{ij}\right)^2}{4 \lambda_{ij}
k_\text{B} T} \right],
\label{equ:marcus}
\end{equation}
%
where $T$ is the temperature, $\lambda_{ij} = \lambda_{ij}^\text{int} + \lambda_{ij}^\text{out}$ is the reorganization energy, which is a sum of intra- and inter-molecular (outer-sphere) contributions, $\Delta E_{ij}$ is the site-energy difference, or driving force, and $J_{ij}$ is the electronic coupling element, or transfer integral~\cite{note_marcus}. A more general, quantum-classical expression for a bimolecular multi-channel rate is derived in the Supporting Information.

All the ingredients entering \equ{marcus} can be calculated using electronic structure techniques, classical simulation methods, or their combination. With the rates at hand, one can study charge transport by solving the differential (master) equation, e.g. by using the kinetic Monte Carlo method which is capable of simulating charge dynamics of non-steady-state systems.

\subsection{Semi-classical bimolecular rate}
The main assumptions in eq.~(\ref{equ:marcus}) are non-adiabaticity (small electronic coupling and  charge transfer between two diabatic, non-interacting states), and harmonic promoting modes, which are treated classically. At ambient conditions, however, the intramolecular promoting mode, which roughly corresponds to C-C bond stretching, has a vibrational energy of $\hbar\omega \approx 0.2\, \unit{eV} \gg k_\text{B}T$ and should be treated quantum-mechanically. The outer-sphere (slow) mode has much lower vibrational energy than the intramolecular promoting mode, and therefore can be treated classically. The weak interaction between molecules also implies that each molecule has its own, practically independent, set of quantum mechanical degrees of freedom.
A rate expression for the aforementioned situation is derived in the supporting information. Numerical estimates show that if  $\lambda_{ij}^\text{int} \approx \lambda_{ij}^\text{out}$ and $|\Delta E_{ij}| \ll \lambda_{ij}^\text{out}$ the rates are similar to those of ~\equ{marcus}. In general, there is no robust method to compute $\lambda_{ij}^\text{out}$~\cite{hoffman_reorganization_1996} and  both reorganization energies are often assumed to be of the same order of magnitude. In this case the second condition also holds, unless there are large differences in electron affinities or ionization potentials of neighboring molecules, e.g. in donor-acceptor blends.