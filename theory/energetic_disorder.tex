\section{Site energies}
\label{sec:site_energies}
A charge transfer reaction between molecules $i$ and $j$ is driven by the site energy\index{site energy} difference, $\Delta E_{ij} = E_i - E_j$. Since the  transfer rate, $\omega_{ij}$, depends exponentially on $\Delta E_{ij}$ (see~\equ{marcus}) it is important to compute its distribution as accurately as possible.  The total site energy difference has contributions due to \slink{ext_field}{externally applied electric field}, \slink{ecoulomb}{electrostatic interactions}, polarization effects, and \slink{internal_energy}{internal energy} differences. In what follows we discuss how to estimate these contributions by making use of first-principles calculations and polarizable force-fields.

\subsection{Externally applied electric field}
\label{sec:ext_field}
The contribution to the total site energy\index{site energy!external field} difference due to an external electric field $\vec{F}$ is given by $\Delta E_{ij}^\text{ext} = q {\vec{F} \cdot \vec{r}_{ij}}$, where $q=\pm e$ is the charge and $\vec{r}_{ij} = \vec{r}_i  - \vec{r}_j $ is a vector connecting molecules $i$ and $j$. For typical distances between small molecules, which are of the order  of $1\,\unit{nm}$, and moderate fields of $F<10^8\,\unit{V/m}$ this term is always smaller than $0.1\, \unit{eV}$.

\subsection{Electrostatic energy}
\label{sec:ecoulomb}
\index{site energy!electrostatic}

Variations of the local electric field can result in large electrostatic contributions to the energetic disorder. Using the atomic partial charges of charged and neutral molecules, $\Delta E_{ij}^\text{el}$ can be computed from the site energies~\cite{kirkpatrick_columnar_2008}
\begin{equation}
E_{i}^\text{el}  = \frac{1}{4 \pi \epsilon_0} \sum_{a_i} \sum_{\substack{b_k   \\ k\neq i }}
\frac{ \left( q^c_{a_i} - q^n_{a_i} \right) q^n_{b_k}}{ \epsilon_\text{s} r_{a_i b_k}} 
\, ,
\label{equ:estatic}
\end{equation}
where $r_{a_i b_k}=|\vec{r}_{a_i} - \vec{r}_{b_k}|$ is the distance between atoms $a_i$ and $b_k$,   $\epsilon_\text{s}$ is the static relative dielectric constant.
%
The first sum extends over all atoms of molecule $i$, for which the site energy is calculated. The second sum reflects interactions with all atoms of neutral molecules $k \ne i$. By using \equ{estatic}, one assumes that the influence of conformational changes on partial charges and changes of the molecular geometry upon charging are small. In order to minimize finite size effects, we do not use spherical cutoff but apply the nearest image convention, that is sum over all neutral molecules in the box after centering the box around the charged molecule. 

The influence of polarization effects on the Coulomb interactions can be taken into account by using a relative dielectric constant in \equ{estatic}. Bulk values of  $\epsilon_\text{s} = 2-5$ for typical organic semiconductors uniformly scale all site energies but are not capable of describing polarization effects on a microscopic level. 
The contribution to $E_i^\text{el}$ from the first coordination shell is then underestimated due to overscreening and, as a result, the site-energy differences become artificially small. Alternatively, one can introduce a phenomenological distance-dependent screening function $\epsilon(r_{a_i b_k})$ in~\equ{estatic}~\cite{nagata_atomistic_2008}
\begin{equation}
\epsilon(r)=\epsilon_{\text{s}} - (\epsilon_{\text{s}} - 1)
\left( 1 + sr + \frac{1}{2}s^2r^2 \right) 
\mathrm{e}^{ -sr}\,,
\label{equ:epss}
\end{equation}
where the parameter $s$ is the inverse screening length. For a monovalent ion in water, for example, $\epsilon_{\text{s}}=80$ and $s=3\,\textrm{nm}^{-1}$~\cite{daggett_molecular_1991}. This screening function ensures that neighboring atoms interact via an unscreened Coulomb potential ($\epsilon \sim 1$) while the electrostatic interaction between atoms at large separations is screened as in the bulk. 

Evaluation of the electrostatic contribution is provided by the \calc{ecoulomb} \calculator. Atomistic partial charges for charged an neutral molecule are taken from files specified in the \xmlsegments files. Note that, in order to speed up calculations for both methods, a cutoff radius (for the molecular centers of mass) can be given in  \xmloptions.

The electrostatic site energies are saved to the \sqlstate file:

{\noindent \small \ctprun \opt \xmloptions  \seg  \xmlsegments \sql  \sqlstate \exe  \calc{ecoulomb} }

\subsection{Internal energy difference}
\label{sec:internal_energy}

The contribution to the site energy difference due to different internal energies\index{site energy!internal} (see \fig{parabolas}) can be written as
\begin{equation}
 \Delta E_{ij}^\text{int}=
\Delta U_i - \Delta U_j = \left( U_{i}^{cC}-U_{i}^{nN}\right) - \left( U_{j}^{cC}-U_{j}^{nN}\right) \, ,
\label{equ:conformational}
\end{equation}
where $U_{i}^{cC(nN)}$ is the total energy of molecule $i$ in the charged (neutral) state and geometry.  $\Delta U_{i}$ corresponds to the adiabatic ionization potential (or electron affinity) of molecule $i$, as shown in~\fig{parabolas}. For one-component systems and negligible conformational changes $ \Delta E_{ij}^\text{int}=0$, while it is significant for donor-acceptor systems.
